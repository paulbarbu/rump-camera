\documentclass[10pt,a4paper,twoside]{article}
\usepackage{amsfonts}
%\usepackage{ICDD}

\makeatletter
    \def\thebibliography#1{\section*{References\@mkboth
      {REFERENCES}{REFERENCES}}\list
      {[\arabic{enumi}]}{\settowidth\labelwidth{[#1]}\leftmargin\labelwidth
    \advance\leftmargin\labelsep
    \usecounter{enumi}}
    \def\newblock{\hskip .11em plus .33em minus .07em}
    \sloppy\clubpenalty4000\widowpenalty4000
    \sfcode`\.=1000\relax}
    \makeatother

%\addtolength{\oddsidemargin}{1.3cm}
%\addtolength{\evensidemargin}{-0.1cm}
\setlength{\topmargin}{-1.5cm}
%\setlength{\textheight}{20cm} \setlength{\textwidth}{13cm}
\setlength{\textheight}{8.875in} \setlength{\textwidth}{6.3in}
\setlength{\oddsidemargin}{.077in}
\setlength{\evensidemargin}{.077in} \thispagestyle{empty}

\begin{document}
%\tableofcontents
\pagestyle{empty}

\def\SHORTTITLE  {An experiment on running surveillance software as a unikernel application}%

\vspace*{3cm}
%\antet
\markboth{\hfill Barbu Paul - Gheorghe}{\hfill
\SHORTTITLE}

\begin{center}
{\Large \bf An experiment on running surveillance software as a unikernel application}
\par\vspace*{0.5cm}
{\bf Barbu Paul - Gheorghe } \\ {\bf Teacher Coordinator: - }
\end{center}

\vspace*{1cm}
%%%%%%%%%%%%%%%%%%%%%%%%%%
%%%%%% PROOF.TEX %%%%%%%%%
%%%%%%%%%%%%%%%%%%%%%%%%%%
\tolerance 10000
\newtheorem{theorem}{Teorem\check{a}}
\newtheorem{lemma}{Lema}
\newtheorem{definition}{Definitie}
\newtheorem{example}{Exemplu}
\newtheorem{xca}{Exercitiu}
\newtheorem{remark}{Observatie}
\newtheorem{proposition}{Propozitie}
\newtheorem{corollary}{Corolar}

\begin{abstract}
The aim of this paper is to explore the recent shift from monolithic applications to using micro-services in a virtualized environment for
serving clients' needs. During the experiment I created an application that runs as a unikernel and manages
video streams from an IP camera, thus running IP camera surveillance software on bare metal hardware and on
type one hypervisors.
\end{abstract}

\section{Introduction}
Since software complexity is increasing with time, teams of software developers and maintainers are trying to
find new ways of keeping this complexity at manageable levels. As a result of this effort are applications
that can be run as the only entity on a system,
the unnecessary parts of the operating systems (virtual memory, process schedulers) are removed and made
optional, only the drivers
remain, these are necessary in order to still be able to use devices
such as disks (be them Solid State Drives or Hard Disk Drives), network interface cards, different input and
output modules (USB, video, etc.).
Once such effort is the \textit{rumpkernel}\footnote{http://rumpkernel.org/}
 whose aim is to run the NetBSD\footnote{https://www.netbsd.org/} kernel in an userspace process.
 This is done with the help of anykernels \footnote{https://wiki.netbsd.org/rumpkernel/}, which are developed
 by the NetBSD team, a project started in order to develop device drivers directly in an userspace process
 \footnote{http://rumpkernel.org/misc/usenix-login-2014/login\_1410\_03\_kantee.pdf}, without needing a virtual machine such as VirtualBox\footnote{https://www.virtualbox.org/} or QEMU\footnote{http://wiki.qemu.org/Main\_Page}.
Later, with the emergence of the unikernels\footnote{HaLVM, MirageOS and others: https://en.wikipedia.org/wiki/Unikernel} this project has evolved into \textit{rumprun}, which allows the programmer to design an application as if it were running directly on hardware. Based on this, I fiddled with the idea of creating surveillance software that runs on bare-metal hardware and in this paper I present the successful materialization of this idea.

\section{Related work \& software}
As far as surveillance software goes, there are plenty of alternatives
\footnote{https://en.wikipedia.org/wiki/IP\_camera\#List\_of\_IP\_camera\_software},
 but no such alternative running as an unikernel exists
 \footnote{as far as my knowledge holds}.
 So this section will proceed with a comparison of unikernel projects.

Unikernels are single address space systems which bundle up an application and a
selection of system components relevant for that particular application into a single image.
The lightweight image can then be run on the cloud or on hardware, depending on what driver
components are available for that particular unikernel system.

One approach for unikernel implementations is to use a single language such as Erlang, Haskell or OCaml and a clean-slate implementation of everything (drivers, network layer, etc).\cite{RumpComparison}

\subsection{HaLVM}

The Haskell Lightweight Virtual Machine describes itself as
"a port of the Glasgow Haskell Compiler tool suite to enable developers to write high-level,
lightweight virtual machines that can run directly on the Xen hypervisor".
\footnote{https://github.com/GaloisInc/HaLVM\#1-what-is-the-halvm}
This essentially means that one can write code in Haskell, compile it, and run it in a hypervised environment
such as Xen. When targeted at the proper applications, the combination of Haskell and the type-safety it provides
together with the isolation provided by Xen can yield profitable results.\footnote{http://galois.com/}

\subsection{MirageOS}
"MirageOS is a library operating system\footnote{Another common name for unikernel} that constructs unikernels for secure, high-performance network applications across a variety of cloud computing and mobile platforms. Code can be developed on a normal OS such as Linux or MacOS X, and then compiled into a fully-standalone, specialised unikernel that runs under the Xen hypervisor."\footnote{https://mirage.io/} This project is similar to the HaLVM, the main difference is the choice of language used for developing the application.

\subsection{Rumprun}
While the other unikernel alternatives require writing the software in a specific language like Haskell or OCaml, the Rumprun unikernel places no such requirement on the programmer. The only requirement of the Rumprun platform is that the code should run on POSIX-like systems, generally speaking this would include Linux, the BSD family and OS X\footnote{https://github.com/rumpkernel/rumprun-packages}. Another advantage of the Rumprun platform would be the fact that over 99\% of the code in the Rumprun stack is used unmodified from NetBSD, hence providing high quality software and code reuse (including drivers, networking stack and security updates).\cite{RumpComparison}

\section{Rump Cameras}
TODO: enumerate the requirements of the software
TODO: dependencies
TODO: debugging is simple: https://github.com/rumpkernel/wiki/wiki/Howto:-Debugging-Rumprun-with-gdb
TODO: memory/disk footprint
TODO: code size
TODO: code online and license


Program listing or program commands in text should be set using
Microsoft Sans Serif, 9pt. as in the following example:
\begin{verbatim}
void main()
{
   printf("Hello World");
}
\end{verbatim}

\section{Future work and conclusions}
Having a single application that runs on a single (low-power) device, with the unnecessary parts removed out
of the kernel (such as the scheduler) and with only the strictly necessary parts included - such as the drivers),
I believe we can write applications that are more environment friendly,
more modular and more robust, being focused on a single, specific task.

Running these unikernel applications in a virtual environment (like Xen) would also be
beneficial to our surrounding medium since, the applications would be started when
needed and stopped at the proper times, thus consuming less energy. \cite{DataCenterEnergyForeCast} \cite{JinWenChen}

This paper successfully demonstrates that surveillance software, with real requirements,
 for IP cameras can be written to run as an unikernel with minimal footprint and dependencies.

\begin{thebibliography}{*}\label{Refences}
\bibitem{RumpComparison}
github.com/rumpkernel, \newblock Info: Comparison of rump kernels with similar technologies,
\newblock https://github.com/rumpkernel/wiki/wiki/Info\%3A-Comparison-of-rump-kernels-with-similar-technologies,
\newblock Accessed 15th of March 2016

\vspace{-7pt}
\bibitem{2}
A. Kantee, J. Cormac, \newblock Rump Kernels No OS? No Problem!,
\newblock http://rumpkernel.org/misc/usenix-login-2014/login\_1410\_03\_kantee.pdf,
\newblock October 2014,
\newblock Accessed 15th of March 2016

\vspace{-7pt}
\bibitem{3}
wikipedia.org, \newblock Unikernel,
\newblock https://en.wikipedia.org/wiki/Unikernel,
\newblock Accessed 14th of March 2016

\vspace{-7pt}
\bibitem{DataCenterEnergyForeCast}
Silicon Valley Leadership Group, Accenture, \newblock
Data Centre Energy Forecast Report-Final Report, \newblock July 2008

\vspace{-7pt}
\bibitem{JinWenChen}
Y. Jin, Y. Wen, Q. Chen,
\newblock Energy Efficiency and Server Virtualization in Data Centers: An Empirical Investigation,
\newblock March 2012

\end{thebibliography}

\vspace*{1cm} {\footnotesize
\begin{tabular*}{16cm}{p{4.2cm}p{4.2cm}p{4.2cm}}
Barbu Paul - Gheorghe & \\
"Lucian Blaga" University of Sibiu & \\
Department of Computer Science and Electrical Engineering & \\
10, Victoriei Bd., Sibiu, 550024, Rom\^ania & \\
ROM\^ANIA & \\
E-mail: \ {\it barbu.paul.gheorghe@gmail.com}&
\end{tabular*}}

\end{document}
